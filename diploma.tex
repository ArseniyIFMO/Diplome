    \documentclass[ 12pt,x11names]{article}
    \usepackage{float}
    \usepackage[utf8x]{inputenc}
    \usepackage[T2A]{fontenc}
    \usepackage[russian]{babel}
    \usepackage{amsfonts}
    \usepackage{amssymb,amsmath,color}

    \usepackage{wrapfig}
    \usepackage{pgfplots}

    \begin{document}
    \section{Введение}
    Состояние магнитной системы описывается
    $E(S) = S^t*A*S + B*S$, где $A$ - матрица $n * n$, $B$ - матрица $1 * n$,
    каждый элемент которой - трехмерный вектор $(x, y, z)$
    А $S$ -- матрица $1*n$, каждый элемент которой - трехмерный вектор $S_i = (x, y, z)$. Эта матрица и описывает состояние магнитной системы.
    Причем, требуется что бы норма каждого из векторов была равна $1$
    ($x^2 + y ^ 2 + z ^ 2 = 1, \|S_{i, j} = 1\|$)
    В нашей задаче
    \begin{equation*}
    A = \left(
    \begin{array}{ccccс}
    -2 & 1 & 0 & 0 & \ldots & 0\\
    1 & -2 & 1 & 0 &\ldots & 0\\
    0 & 1 & -2 & 1 &\ldots & 0\\
    \vdots &\vdots &\vdots &  &\ddots & \vdots\\
    0 & 0 &  0&   0 &\ldots & -2
    \end{array}
    \right)
    \end{equation*}\\
    В конечном итоге, нам требуется минимизировать данный функционал.\\
    Записать энергию $E(S)$ можно тогда по другому.\\
    $x = (1,0, 0), y = (0,1, 0), z =(0, 0 , 1)$\\
    $E(S) = \displaystyle{\sum_{i, j}} J* (S_{i - 1, j} + S_{i + 1, j} + S_{i, j - 1} + S_{i, j + 1}) \cdot S_{i,j} + \\
    D * (-S_{i - 1, j} \times x  + S_{i + 1, j} \times x - S_{i, j - 1} \times y - S_{i, j + 1} \times y)
    + \\
    K * (z \cdot S_{i,j}) ^ 2$\\
    \section{Методы решения}
    \subsection{Метод 1}
    Первый метод которым можно воспользоваться  - градиентный спуск.
    $\nabla E(S)_{i, j} = J * (S_{i + 1,j} + S_{i - 1,j} + S_{i, j + 1} + S_{i, j - 1})
        + \\ D * ( (S_{i + 1,j} \times x) + (S_{i, j + 1} \times y) -  (S_{i - 1, j} \times x) - (S_{i}{j - 1} \times y) )
        + 2 * z * K * (z \cdot S[i][j])$\\
    Будем  итерироваться как в градиентнтом спуске $S_{i, j} = S_{i, j} - \alpha * \nabla E(S)_{i, j}$.\\
    Однако, очевидно, после такого шага может  нарушится условие $\|S_{i, j} = 1\|$.\\
    Тогда, будем действовать следующим образом. Сначала \\$S_{i, j} = S_{i, j} - \alpha * \nabla E(S)_{i, j}$, а потом отнормируем каждый элемент, то есть $S_{i, j} = \frac{S_{i, j}}{\|S_{i, j}\|}$.\\
    В этом варианте берется $\alpha = 0.001$\\
    \subsection{Метод 2}
    Воспользуемся методом сопряженных градиентов.\\
    Сперва опробуем метод Флетчера - Риза
    $S^k_{i, j} = S^{k-1}_{i, j} + \alpha * \Delta^k $, где $\Delta^k = -\nabla S_{i, j} + \omega * \Delta ^  {k-1} $,
    $\omega  = \frac{\| \nabla S^k \|}{\| \nabla S^{k-1} \|}$


    \subsection{Метод 3}
    Для дальнейшего ускорения можно использовать multigrid.
    У градиентного спуска имеется некоторая начальная точка, она может быть довольно далека от оптимальной.
    Что бы начинать в точке ближе к оптимальной и, соответственно, что бы алгоритм сходиля быстрее и  используется multigrid.\\

    \end{document}
