    \documentclass[ 12pt,x11names]{article}
    \usepackage{float}
    \usepackage[utf8x]{inputenc}
    \usepackage[T2A]{fontenc}
    \usepackage[russian]{babel}
    \usepackage{amsfonts}
    \usepackage{amssymb,amsmath,color}
    \usepackage[14pt]{extsizes}
    \usepackage{wrapfig}
    \usepackage{pgfplots}
    \usepackage{indentfirst}

% Default fixed font does not support bold face
\DeclareFixedFont{\ttb}{T1}{txtt}{bx}{n}{12} % for bold
\DeclareFixedFont{\ttm}{T1}{txtt}{m}{n}{12}  % for normal

% Custom colors
\usepackage{color}
\definecolor{deepblue}{rgb}{0,0,0.5}
\definecolor{deepred}{rgb}{0.6,0,0}
\definecolor{deepgreen}{rgb}{0,0.5,0}

\usepackage{listings}

% Python style for highlighting
\newcommand\pythonstyle{\lstset{
language=Python,
basicstyle=\ttm,
otherkeywords={self},             % Add keywords here
keywordstyle=\ttb\color{deepblue},
emph={MyClass,__init__},          % Custom highlighting
emphstyle=\ttb\color{deepred},    % Custom highlighting style
stringstyle=\color{deepgreen},
frame=tb,                         % Any extra options here
showstringspaces=false            %
}}


% Python environment
\lstnewenvironment{python}[1][]
{
\pythonstyle
\lstset{#1}
}
{}

% Python for external files
\newcommand\pythonexternal[2][]{{
\pythonstyle
\lstinputlisting[#1]{#2}}}

% Python for inline
\newcommand\pythoninline[1]{{\pythonstyle\lstinline!#1!}}
    \begin{document}
    \section{Введение}\\
         Характеризация магнитных материалов начинается с  поиска метастабильного состояния системы.
    В данной работе будет решаться задача поиска такого состояния.\\
        Метастабильное состояние - это состояние системы, в котором энергия минимальна, и, соответственно, в работе требуется находить минимальную энергию некоторой магнитной системы.\\
    Энергия магнитной системы выражается некоторым функционалом, зависящим от положения каждого из скирмионов, причем на положения скирмионов накладываются некоторые ограничения. \\Соответственно, в данной работе будет произведено иследование различных методов поиска минимума таких функционалов, так как в силу ограничений, накладываемых на скирмионы, использование стандартных алгоритмов невозможно.\\
    В данной работе я не только приведу пример алгоритма, решающего данную задачу, но и   рассмотрю различне методы выполнения данной задачи, реализую их на языке Python, сравню их скорость сходимости, сделаю вывод о их работоспособности в условиях данной задачи, времени работы и так далее.\\
    Данная задача очень важна в теоретической физике, особенно важна скорость сходимости, так как в реальных физических системах количество частиц может быть очень велико, и ассимтотика очень важна прирешении данной задачи.\\
    Я собираюсь использовать градиентный спуск, рассмотрю скорость и точность сходимости в зависимости от параметров, различные его вариации, а так же многосеточные методы для решения данной задачи.\\
     Практическая значимость полученных результатов состоит в возможности экономии времени других исследователей при решении данной задачи, мы найдем какой алгоритм поиска метастабильного состояния является наиболее оптимальным и точным, и, соответственно, убережем других исседователей от выборов неправильных или неэффективных инструментов.\\

    \newpage
    \section{Теоретическая часть работы}
    Состояние магнитной системы описывается
    $E(S) = S^t*A*S + B*S$, где $A$ - матрица $n * n$, $B$ - матрица $1 * n$,
    каждый элемент которой - трехмерный вектор $(x, y, z)$
    А $S$ -- матрица $1*n$, каждый элемент которой - трехмерный вектор $S_i = (x, y, z)$ - вектор спина $i$-го скирмиона. \\
        Эта матрица и описывает состояние магнитной системы.
    Причем, требуется что бы норма каждого из векторов была равна $1$.
    ($x^2 + y ^ 2 + z ^ 2 = 1, \|S_{i, j} = 1\|$)\\
    В нашей задаче:
    \begin{equation*}
    A = \left(
    \begin{array}{ccccс}
    -2 & 1 & 0 & 0 & \ldots & 0\\
    1 & -2 & 1 & 0 &\ldots & 0\\
    0 & 1 & -2 & 1 &\ldots & 0\\
    \vdots &\vdots &\vdots &  &\ddots & \vdots\\
    0 & 0 &  0&   0 &\ldots & -2
    \end{array}
    \right)
    \end{equation*}\\
    В конечном итоге, нам требуется минимизировать данный функционал.\\
    Записать энергию $E(S)$ можно тогда по другому.\\
    $x = (1,0, 0), y = (0,1, 0), z =(0, 0 , 1)$\\
    $E(S) = \displaystyle{\sum_{i, j}} J* (S_{i - 1, j} + S_{i + 1, j} + S_{i, j - 1} + S_{i, j + 1}) \cdot S_{i,j} + \\
    D * (-S_{i - 1, j} \times x  + S_{i + 1, j} \times x - S_{i, j - 1} \times y - S_{i, j + 1} \times y)
    + \\
    K * (z \cdot S_{i,j}) ^ 2$,\\
    где
    $\displaystyle{\sum_{i, j}} J* (S_{i - 1, j} + S_{i + 1, j} + S_{i, j - 1} + S_{i, j + 1}) \cdot S_{i,j}$ -- энергия обмена, а $D * (-S_{i - 1, j} \times x  + S_{i + 1, j} \times x - S_{i, j - 1} \times y - S_{i, j + 1} \times y)$  -- взаимодействие Джелошински - Мория.\\
    Обычно, таки задачи решаются градиентным спуском, однако, в данной ситуации он не применим напрямую из за ограничений на норму спинов.
    Далее,мы рассмотрим несколько алгоритмов, которыми мы попытаемся решить данную задачу.
    \newpage
    \section{Методы решения}
    \subsection{Метод 1}
    Первый метод которым можно воспользоваться  - градиентный спуск.
    $\nabla E(S)_{i, j} = J * (S_{i + 1,j} + S_{i - 1,j} + S_{i, j + 1} + S_{i, j - 1})
        + \\ D * ( (S_{i + 1,j} \times x) + (S_{i, j + 1} \times y) -  (S_{i - 1, j} \times x) - (S_{i}{j - 1} \times y) )
        + 2 * z * K * (z \cdot S[i][j])$\\
    Будем  итерироваться как в градиентнтом спуске $S_{i, j} = S_{i, j} - \alpha * \nabla E(S)_{i, j}$.\\
    Однако, очевидно, после такого шага может  нарушится условие $\|S_{i, j} = 1\|$.\\
    Тогда, будем действовать следующим образом. Сначала \\$S_{i, j} = S_{i, j} - \alpha * \nabla E(S)_{i, j}$, а потом отнормируем каждый элемент, то есть $S_{i, j} = \frac{S_{i, j}}{\|S_{i, j}\|}$.\\
    В этом варианте берется $\alpha = 0.001$\\
    Сущетвует несколько способов задать критерий остановки: например, сделать фиксированное число итераций, либо делать итерации пока !!!!!
    \subsection{Метод 2}
    Воспользуемся методом сопряженных градиентов.\\
    Сперва опробуем метод Флетчера - Риза
    $S^k_{i, j} = S^{k-1}_{i, j} + \alpha * \Delta^k $, где $\Delta^k = -\nabla S_{i, j} + \omega * \Delta ^  {k-1} $,
    $\omega  = \frac{\| \nabla S^k \|^2}{\| \nabla S^{k-1} \|^2}$\\
    Будем считать, что $\Delta^0=(0,0,0)$
    \subsection{Метод 3}
    Попробуем для решения данной задачи использовать метод второго порядка. Например, метод Ньютона.

    \subsection{Метод 4}
    Для дальнейшего ускорения можно использовать multigrid.
    У градиентного спуска имеется некоторая начальная точка, она может быть довольно далека от оптимальной.
    Что бы начинать в точке ближе к оптимальной и, соответственно, что бы алгоритм сходиля быстрее и  используется multigrid.\\
    Алгоритм
    \newpage
    \section{Практическая часть}
    \subsection{Метод 1}
    Приведу реализацию алгоритма на питоне и  сравним их быстродействие.
    Реализация с фиксированным $\alpha$\\
\begin{python}
    def grad(i, j):
        tmp = J * (S[i + 1][j] + S[i - 1][j] +
        +S[i][j + 1] + S[i][j - 1])
        tmp2 = D * (np.cross(S[i + 1][j], x)
        +np.cross(S[i][j + 1], y)
        - np.cross(S[i - 1][j], x)
        - np.cross(S[i][j - 1], y))
        tmp3 = 2 * z * K * np.dot  (z, S[i][j]).item()
        res = - tmp + tmp2 - tmp3
        return res

    def E():
        res = 0
        for i in range(1, SX + 1):
            for j in range(1, SY + 1):
                tmp = J * np.dot((S[i + 1][j] + S[i - 1][j] +
                S[i][j + 1] + S[i][j - 1]), S[i][j])
                tmp2 = D * (
                + np.dot(np.cross(S[i + 1][j], S[i][j]), x)
                + np.dot(np.cross(S[i][j + 1], S[i][j]), y)
                - np.dot(np.cross(S[i - 1][j], S[i][j]), x)
                - np.dot(np.cross(S[i][j - 1], S[i][j]), y))
                tmp3 = K * (np.dot(z, S[i][j]) ** 2)
                res = res - tmp / 2 - tmp2 / 2 - tmp3
        return res

    def normalize():
    for i in range(1, SX + 1):
        for j in range(1, SY + 1):
            norm = S[i][j][0] * S[i][j][0] +
                   S[i][j][1] * S[i][j][1] +
                   S[i][j][2] * S[i][j][2]
            norm = np.sqrt(norm)
            S[i][j][0] = S[i][j][0] / norm
            S[i][j][1] = S[i][j][1] / norm
            S[i][j][2] = S[i][j][2] / norm


    SX = 4
    SY = 5
    x = np.array([1.0, 0.0, 0.0])
    y = np.array([0.0, 1.0, 0.0])
    z = np.array([0.0, 0.0, 1.0])
    S = np.random.randn(SX + 2, SY + 2, 3)
    for i in range(0, SX + 2):
        S[i][0] = np.array([0, 0, 0])
        S[i][SY + 1] = np.array([0, 0, 0])

    for i in range(0, SY + 2):
        S[0][i] = np.array([0, 0, 0])
        S[SX + 1][i] = np.array([0, 0, 0])

    K = 0.4
    D = 0.35
    J = 1
    step = 0.1

    for k in range(0, 10000):
        newS = np.zeros_like(S)
        maxNorm = 0
        for i in range(1, SX + 1):
            for j in range(1, SY + 1):
                g = grad(i,  j)
                projGradOnS = np.dot(S[i][j], g)
                g = g - projGradOnS * S[i][j]
                maxNorm = np.maximum(maxNorm, np.linalg.norm(g))
                newS[i][j] = S[i][j] - step * grad(i, j)

        S = newS
        normalize()

print(E())
\end{python}
\subsection{Метод 2}
 Приведу реализацию метода сопряженных градиентов Флетчера - Ривза
 \begin{python}
 import numpy as np
SX = 4
SY = 5
x = np.array([1.0, 0.0, 0.0])
y = np.array([0.0, 1.0, 0.0])
z = np.array([0.0, 0.0, 1.0])
S = np.random.randn(SX + 2, SY + 2, 3)
for i in range(0, SX + 2):
    S[i][0] = np.array([0, 0, 0])
    S[i][SY + 1] = np.array([0, 0, 0])

for i in range(0, SY + 2):
    S[0][i] = np.array([0, 0, 0])
    S[SX + 1][i] = np.array([0, 0, 0])

K = 0.4
D = 0.35
J = 1
step = 0.1

def normalPrintS():
    for i in range(0, SX + 2):
        print()
        for j in range(0, SY + 2):
            print(S[i][j], end=" ")
    print()


def grad(i, j):
    tmp = J * (S[i + 1][j] + S[i - 1][j] +
        S[i][j + 1] + S[i][j - 1])
    tmp2 = D * (np.cross(S[i + 1][j], x) +
    np.cross(S[i][j + 1], y) -
    np.cross(S[i - 1][j], x) - np.cross(S[i][j - 1], y))
    tmp3 = 2 * z * K * np.dot(z, S[i][j]).item()
    res = - tmp + tmp2 - tmp3
    return res


def E():
    res = 0
    for i in range(1, SX + 1):
        for j in range(1, SY + 1):
            tmp = J * np.dot((S[i + 1][j] + S[i - 1][j] + S[i][j + 1] + S[i][j - 1]), S[i][j])
            tmp2 = D * (+ np.dot(np.cross(S[i + 1][j], S[i][j]), x)
                        + np.dot(np.cross(S[i][j + 1], S[i][j]), y)
                        - np.dot(np.cross(S[i - 1][j], S[i][j]), x)
                        - np.dot(np.cross(S[i][j - 1], S[i][j]), y))
            tmp3 = K * (np.dot(z, S[i][j]) ** 2)
            res = res - tmp / 2 - tmp2 / 2 - tmp3
    return res

def E2():
    res = 0
    for i in range(1, SX + 1):
        for j in range(1, SY + 1):
            res += np.dot(S[i][j], grad(i,j))
    return res * 0.5

def normalize():
    for i in range(1, SX + 1):
        for j in range(1, SY + 1):
            norm = S[i][j][0] * S[i][j][0] + S[i][j][1] * S[i][j][1]
            + S[i][j][2] * S[i][j][2]
            norm = np.sqrt(norm)
            S[i][j][0] = S[i][j][0] / norm
            S[i][j][1] = S[i][j][1] / norm
            S[i][j][2] = S[i][j][2] / norm

normalize()

prev_grad =  np.random.randn(SX + 2, SY + 2, 3)
cur_grad = np.random.randn(SX + 2, SY + 2, 3)
for i in range(1, SX + 1):
    for j in range(1, SY + 1):
        prev_grad[i][j] = grad(i, j)

for k in range(0, 10000):
    newS = np.zeros_like(S)
    maxNorm = 0
    sum_of_prev_grad = 0
    sum_of_grad  =  0


    for i in range(1, SX + 1):
        for j in range(1, SY + 1):
            sum_of_prev_grad = sum_of_prev_grad +
            np.dot(prev_grad[i][j], prev_grad[i][j])

    for i in range(1, SX + 1):
        for j in range(1, SY + 1):
            sum_of_grad = sum_of_grad +
            np.dot(grad(i, j), grad(i, j))

    omega = sum_of_grad / sum_of_prev_grad

    for i in range(1, SX + 1):
        for j in range(1, SY + 1):
            g = grad(i,  j)
            projGradOnS = np.dot(S[i][j], g)
            g = g - projGradOnS * S[i][j]
            maxNorm = np.maximum(maxNorm, np.linalg.norm(g))
            if(k == 0):
                prev_grad[i][j] = grad(i, j)
                newS[i][j] = S[i][j] -
                step * (grad(i, j))
            else:
                newS[i][j] = S[i][j] - step * (grad(i, j))
                + ( step * omega *  prev_grad[i][j] )
                prev_grad[i][j] = grad(i, j) +
                prev_grad[i][j] * omega
    S = newS
    normalize()
    print(E(), maxNorm)

print(E())
 \end{python}
 заметим, что в этой реализации алгоритма многократно пересчитывается $\nabla S_{i, j}$. Тогда можно не каждый раз вычислять ее, а  просто сохранять значение в массиве, что определенно даст ускорение.


 \begin{python}
 import numpy as np
SX = 4
SY = 5
x = np.array([1.0, 0.0, 0.0])
y = np.array([0.0, 1.0, 0.0])
z = np.array([0.0, 0.0, 1.0])
S = np.random.randn(SX + 2, SY + 2, 3)
for i in range(0, SX + 2):
    S[i][0] = np.array([0, 0, 0])
    S[i][SY + 1] = np.array([0, 0, 0])

for i in range(0, SY + 2):
    S[0][i] = np.array([0, 0, 0])
    S[SX + 1][i] = np.array([0, 0, 0])

K = 0.4
D = 0.35
J = 1
step = 0.1

def normalPrintS():
    for i in range(0, SX + 2):
        print()
        for j in range(0, SY + 2):
            print(S[i][j], end=" ")
    print()


def grad(i, j):
    tmp = J * (S[i + 1][j] + S[i - 1][j] +
        S[i][j + 1] + S[i][j - 1])
    tmp2 = D * (np.cross(S[i + 1][j], x) +
    np.cross(S[i][j + 1], y) -
    np.cross(S[i - 1][j], x) - np.cross(S[i][j - 1], y))
    tmp3 = 2 * z * K * np.dot(z, S[i][j]).item()
    res = - tmp + tmp2 - tmp3
    return res


def E():
    res = 0
    for i in range(1, SX + 1):
        for j in range(1, SY + 1):
            tmp = J * np.dot((S[i + 1][j] + S[i - 1][j] + S[i][j + 1] + S[i][j - 1]), S[i][j])
            tmp2 = D * (+ np.dot(np.cross(S[i + 1][j], S[i][j]), x)
                        + np.dot(np.cross(S[i][j + 1], S[i][j]), y)
                        - np.dot(np.cross(S[i - 1][j], S[i][j]), x)
                        - np.dot(np.cross(S[i][j - 1], S[i][j]), y))
            tmp3 = K * (np.dot(z, S[i][j]) ** 2)
            res = res - tmp / 2 - tmp2 / 2 - tmp3
    return res

def E2():
    res = 0
    for i in range(1, SX + 1):
        for j in range(1, SY + 1):
            res += np.dot(S[i][j], grad(i,j))
    return res * 0.5

def normalize():
    for i in range(1, SX + 1):
        for j in range(1, SY + 1):
            norm = S[i][j][0] * S[i][j][0] + S[i][j][1] * S[i][j][1]
            + S[i][j][2] * S[i][j][2]
            norm = np.sqrt(norm)
            S[i][j][0] = S[i][j][0] / norm
            S[i][j][1] = S[i][j][1] / norm
            S[i][j][2] = S[i][j][2] / norm

normalize()

prev_grad =  np.random.randn(SX + 2, SY + 2, 3)
cur_grad = np.random.randn(SX + 2, SY + 2, 3)
for i in range(1, SX + 1):
    for j in range(1, SY + 1):
        prev_grad[i][j] = grad(i, j)
for i in range(1, SX + 1):
    for j in range(1, SY + 1):
        cur_grad[i][j] = grad(i,  j)
for k in range(0, 10000):
    newS = np.zeros_like(S)
    maxNorm = 0
    sum_of_prev_grad = 0
    sum_of_grad  =  0


    for i in range(1, SX + 1):
        for j in range(1, SY + 1):
            sum_of_prev_grad = sum_of_prev_grad +
            np.dot(prev_grad[i][j], prev_grad[i][j])

    for i in range(1, SX + 1):
        for j in range(1, SY + 1):
            sum_of_grad = sum_of_grad +
            np.dot( cur_grad[i][j] ,  cur_grad[i][j] )

    omega = sum_of_grad / sum_of_prev_grad

    for i in range(1, SX + 1):
        for j in range(1, SY + 1):
            g = grad(i,  j)
            projGradOnS = np.dot(S[i][j], g)
            g = g - projGradOnS * S[i][j]
            maxNorm = np.maximum(maxNorm, np.linalg.norm(g))
            if(k == 0):
                prev_grad[i][j] = grad(i, j)
                newS[i][j] = S[i][j] -
                step * (grad(i, j))
            else:
                newS[i][j] = S[i][j] - step * ( cur_grad[i][j] )
                + ( step * omega *  prev_grad[i][j] )
                prev_grad[i][j] =  cur_grad[i][j]  +
                prev_grad[i][j] * omega
    S = newS
    normalize()
    print(E(), maxNorm)

print(E())
 \end{python}
 \subsection{Метод 2.1}
 Можно попробоватьреализовать обычный градиентный спуск, но не с фиксированным $\alpha$, а с $\alpha = \frac{\| \nabla S^k \|^2}{\| \nabla S^{k-1} \|^2}$\\
  \begin{python}

SX = 4
SY = 5
x = np.array([1.0, 0.0, 0.0])
y = np.array([0.0, 1.0, 0.0])
z = np.array([0.0, 0.0, 1.0])
S = np.random.randn(SX + 2, SY + 2, 3)
for i in range(0, SX + 2):
    S[i][0] = np.array([0, 0, 0])
    S[i][SY + 1] = np.array([0, 0, 0])

for i in range(0, SY + 2):
    S[0][i] = np.array([0, 0, 0])
    S[SX + 1][i] = np.array([0, 0, 0])

K = 0.4
D = 0.35
J = 1
step = 0.1

def normalPrintS():
    for i in range(0, SX + 2):
        print()
        for j in range(0, SY + 2):
            print(S[i][j], end=" ")
    print()


def grad(i, j):
    tmp = J * (S[i + 1][j] + S[i - 1][j] + S[i][j + 1] + S[i][j - 1])
    tmp2 = D * (np.cross(S[i + 1][j], x) + np.cross(S[i][j + 1], y) - np.cross(S[i - 1][j], x) - np.cross(S[i][j - 1], y))
    tmp3 = 2 * z * K * np.dot(z, S[i][j]).item()
    res = - tmp + tmp2 - tmp3
    return res


def E():
    res = 0
    for i in range(1, SX + 1):
        for j in range(1, SY + 1):
            tmp = J * np.dot((S[i + 1][j] + S[i - 1][j] + S[i][j + 1] + S[i][j - 1]), S[i][j])
            tmp2 = D * (+ np.dot(np.cross(S[i + 1][j], S[i][j]), x)
                        + np.dot(np.cross(S[i][j + 1], S[i][j]), y)
                        - np.dot(np.cross(S[i - 1][j], S[i][j]), x)
                        - np.dot(np.cross(S[i][j - 1], S[i][j]), y))
            tmp3 = K * (np.dot(z, S[i][j]) ** 2)
            res = res - tmp / 2 - tmp2 / 2 - tmp3
    return res

def E2():
    res = 0
    for i in range(1, SX + 1):
        for j in range(1, SY + 1):
            res += np.dot(S[i][j], grad(i,j))
    return res * 0.5

def normalize():
    for i in range(1, SX + 1):
        for j in range(1, SY + 1):
            norm = S[i][j][0] * S[i][j][0] + S[i][j][1] * S[i][j][1] + S[i][j][2] * S[i][j][2]
            norm = np.sqrt(norm)
            S[i][j][0] = S[i][j][0] / norm
            S[i][j][1] = S[i][j][1] / norm
            S[i][j][2] = S[i][j][2] / norm

normalize()

prev_grad =  np.random.randn(SX + 2, SY + 2, 3)
cur_grad = np.random.randn(SX + 2, SY + 2, 3)
for i in range(1, SX + 1):
    for j in range(1, SY + 1):
        prev_grad[i][j] = grad(i, j)

for k in range(0, 10000):
    newS = np.zeros_like(S)
    maxNorm = 0
    sum_of_prev_grad = 0
    sum_of_grad  =  0
    for i in range(1, SX + 1):
        for j in range(1, SY + 1):
            cur_grad[i][j] = grad(i,  j)

    for i in range(1, SX + 1):
        for j in range(1, SY + 1):
            sum_of_prev_grad = sum_of_prev_grad + np.dot(prev_grad[i][j], prev_grad[i][j])

    for i in range(1, SX + 1):
        for j in range(1, SY + 1):
            sum_of_grad = sum_of_grad + np.dot(cur_grad[i][j], cur_grad[i][j])

    alpha = sum_of_grad / sum_of_prev_grad

    for i in range(1, SX + 1):
        for j in range(1, SY + 1):
            g = grad(i,  j)
            projGradOnS = np.dot(S[i][j], g)
            g = g - projGradOnS * S[i][j]
            maxNorm = np.maximum(maxNorm, np.linalg.norm(g))
            prev_grad[i][j] =grad(i, j)
            newS[i][j] = S[i][j] - alpha * grad(i, j)

    S = newS
    normalize()
    print(E(), maxNorm)

print(E())

 \end{python}
\newpage
\section{Сравнение быстродействия методов}
 Сравним быстродействие данных методов:\\
 \begin{tikzpicture}
\begin{axis}[
    title={Метод 1},
    xlabel={N},
    ylabel={Solubility [g per 100 g water]},
    xmin=0, xmax=100,
    ymin=0, ymax=120,
    xtick={0,20,40,60,80,100},
    ytick={0,20,40,60,80,100,120},
    legend pos=north west,
    ymajorgrids=true,
    grid style=dashed,
]

\addplot[
    color=blue,
    mark=square,
    ]
    coordinates {
    (0,23.1)(10,27.5)(20,32)(30,37.8)(40,44.6)(60,61.8)(80,83.8)(100,114)
    };
    \legend{CuSO$_4\cdot$5H$_2$O}

\end{axis}
\end{tikzpicture}\\
 \begin{tikzpicture}
\begin{axis}[
    title={Метод 2},
    xlabel={N},
    ylabel={Solubility [g per 100 g water]},
    xmin=0, xmax=100,
    ymin=0, ymax=120,
    xtick={0,20,40,60,80,100},
    ytick={0,20,40,60,80,100,120},
    legend pos=north west,
    ymajorgrids=true,
    grid style=dashed,
]

\addplot[
    color=blue,
    mark=square,
    ]
    coordinates {
    (0,23.1)(10,27.5)(20,32)(30,37.8)(40,44.6)(60,61.8)(80,83.8)(100,114)
    };
    \legend{CuSO$_4\cdot$5H$_2$O}

\end{axis}
\end{tikzpicture}\\

 \begin{tikzpicture}
 \begin{axis}[
    title={Метод 3},
    xlabel={N},
    ylabel={Solubility [g per 100 g water]},
    xmin=0, xmax=100,
    ymin=0, ymax=120,
    xtick={0,20,40,60,80,100},
    ytick={0,20,40,60,80,100,120},
    legend pos=north west,
    ymajorgrids=true,
    grid style=dashed,
]

\addplot[
    color=blue,
    mark=square,
    ]
    coordinates {
    (0,23.1)(10,27.5)(20,32)(30,37.8)(40,44.6)(60,61.8)(80,83.8)(100,114)
    };
    \legend{CuSO$_4\cdot$5H$_2$O}

\end{axis}
\end{tikzpicture}\\

 \begin{tikzpicture}
\begin{axis}[
    title={Метод 4},
    xlabel={N},
    ylabel={Solubility [g per 100 g water]},
    xmin=0, xmax=100,
    ymin=0, ymax=120,
    xtick={0,20,40,60,80,100},
    ytick={0,20,40,60,80,100,120},
    legend pos=north west,
    ymajorgrids=true,
    grid style=dashed,
]

\addplot[
    color=blue,
    mark=square,
    ]
    coordinates {
    (0,23.1)(10,27.5)(20,32)(30,37.8)(40,44.6)(60,61.8)(80,83.8)(100,114)
    };
    \legend{CuSO$_4\cdot$5H$_2$O}

\end{axis}
\end{tikzpicture}\\

 \begin{tikzpicture}
\begin{axis}[
    title={Метод 5},
    xlabel={N},
    ylabel={Solubility [g per 100 g water]},
    xmin=0, xmax=100,
    ymin=0, ymax=120,
    xtick={0,20,40,60,80,100},
    ytick={0,20,40,60,80,100,120},
    legend pos=north west,
    ymajorgrids=true,
    grid style=dashed,
]

\addplot[
    color=blue,
    mark=square,
    ]
    coordinates {
    (0,23.1)(10,27.5)(20,32)(30,37.8)(40,44.6)(60,61.8)(80,83.8)(100,114)
    };
    \legend{CuSO$_4\cdot$5H$_2$O}

\end{axis}
\end{tikzpicture}\\

%Here ends the furst plot

 \newpage
    \section{Заключение}
    Как можно заметить, все перечисленные мною методы решают поставленную задачу, с той или иной скоростью. Соответственное, при решении данной задачи можно пользоваться любым из них.
    \end{document}
